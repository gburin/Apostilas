\documentclass[]{article}
\usepackage{lmodern}
\usepackage{amssymb,amsmath}
\usepackage{ifxetex,ifluatex}
\usepackage{fixltx2e} % provides \textsubscript
\ifnum 0\ifxetex 1\fi\ifluatex 1\fi=0 % if pdftex
  \usepackage[T1]{fontenc}
  \usepackage[utf8]{inputenc}
\else % if luatex or xelatex
  \ifxetex
    \usepackage{mathspec}
    \usepackage{xltxtra,xunicode}
  \else
    \usepackage{fontspec}
  \fi
  \defaultfontfeatures{Mapping=tex-text,Scale=MatchLowercase}
  \newcommand{\euro}{€}
\fi
% use upquote if available, for straight quotes in verbatim environments
\IfFileExists{upquote.sty}{\usepackage{upquote}}{}
% use microtype if available
\IfFileExists{microtype.sty}{%
\usepackage{microtype}
\UseMicrotypeSet[protrusion]{basicmath} % disable protrusion for tt fonts
}{}
\usepackage[margin=1in]{geometry}
\usepackage{color}
\usepackage{fancyvrb}
\newcommand{\VerbBar}{|}
\newcommand{\VERB}{\Verb[commandchars=\\\{\}]}
\DefineVerbatimEnvironment{Highlighting}{Verbatim}{commandchars=\\\{\}}
% Add ',fontsize=\small' for more characters per line
\usepackage{framed}
\definecolor{shadecolor}{RGB}{248,248,248}
\newenvironment{Shaded}{\begin{snugshade}}{\end{snugshade}}
\newcommand{\KeywordTok}[1]{\textcolor[rgb]{0.13,0.29,0.53}{\textbf{{#1}}}}
\newcommand{\DataTypeTok}[1]{\textcolor[rgb]{0.13,0.29,0.53}{{#1}}}
\newcommand{\DecValTok}[1]{\textcolor[rgb]{0.00,0.00,0.81}{{#1}}}
\newcommand{\BaseNTok}[1]{\textcolor[rgb]{0.00,0.00,0.81}{{#1}}}
\newcommand{\FloatTok}[1]{\textcolor[rgb]{0.00,0.00,0.81}{{#1}}}
\newcommand{\CharTok}[1]{\textcolor[rgb]{0.31,0.60,0.02}{{#1}}}
\newcommand{\StringTok}[1]{\textcolor[rgb]{0.31,0.60,0.02}{{#1}}}
\newcommand{\CommentTok}[1]{\textcolor[rgb]{0.56,0.35,0.01}{\textit{{#1}}}}
\newcommand{\OtherTok}[1]{\textcolor[rgb]{0.56,0.35,0.01}{{#1}}}
\newcommand{\AlertTok}[1]{\textcolor[rgb]{0.94,0.16,0.16}{{#1}}}
\newcommand{\FunctionTok}[1]{\textcolor[rgb]{0.00,0.00,0.00}{{#1}}}
\newcommand{\RegionMarkerTok}[1]{{#1}}
\newcommand{\ErrorTok}[1]{\textbf{{#1}}}
\newcommand{\NormalTok}[1]{{#1}}
\usepackage{longtable,booktabs}
\ifxetex
  \usepackage[setpagesize=false, % page size defined by xetex
              unicode=false, % unicode breaks when used with xetex
              xetex]{hyperref}
\else
  \usepackage[unicode=true]{hyperref}
\fi
\hypersetup{breaklinks=true,
            bookmarks=true,
            pdfauthor={Laboratório de Macroevolução e Macroecologia},
            pdftitle={Apostila SSH},
            colorlinks=true,
            citecolor=blue,
            urlcolor=blue,
            linkcolor=magenta,
            pdfborder={0 0 0}}
\urlstyle{same}  % don't use monospace font for urls
\setlength{\parindent}{0pt}
\setlength{\parskip}{6pt plus 2pt minus 1pt}
\setlength{\emergencystretch}{3em}  % prevent overfull lines
\setcounter{secnumdepth}{5}

%%% Use protect on footnotes to avoid problems with footnotes in titles
\let\rmarkdownfootnote\footnote%
\def\footnote{\protect\rmarkdownfootnote}

%%% Change title format to be more compact
\usepackage{titling}
\setlength{\droptitle}{-2em}
  \title{Apostila SSH}
  \pretitle{\vspace{\droptitle}\centering\huge}
  \posttitle{\par}
  \author{Laboratório de Macroevolução e Macroecologia}
  \preauthor{\centering\large\emph}
  \postauthor{\par}
  \predate{\centering\large\emph}
  \postdate{\par}
  \date{Maio/2015}




\begin{document}

\maketitle


{
\hypersetup{linkcolor=black}
\setcounter{tocdepth}{3}
\tableofcontents
}
\section{Introdução}\label{introducao}

\subsection{O que é SSH?}\label{o-que-e-ssh}

SSH é a sigla para `Secure SHell', e é basicamente um protocolo de rede
que permite computadores se conectarem via rede de forma a permitir que
um computador execute comandos em unidades remotas de maneira segura
(criptografada). Essa segurança é garantida por uma chave de segurança
que é armazenada tanto em no computador local quanto na máquina remota.
Além disso, na maioria das vezes esse acesso é somente autorizado
através de senhas de acesso.

\subsection{Acesso}\label{acesso}

O acesso básico a uma máquina remota via SSH é bastante simples:

\begin{Shaded}
\begin{Highlighting}[]
\NormalTok{$}\StringTok{ }\NormalTok{ssh usuario@ip}
\end{Highlighting}
\end{Shaded}

No entanto, alguns argumentos podem ser usados conjuntamente com esse
comando básico quando a tarefa requer especificações extras. Uma lista
destes argumentos e suas funções pode ser facilmente obtida usando o
comando

\begin{verbatim}
$ man ssh
\end{verbatim}

que abre o manual deste programa. (Esta sintaxe pode ser usada para
qualquer programa na linha de comando. \emph{Ex. man cp, man emacs,
etc.})

Apesar de vários desses argumentos serem importantes, para nosso acesso
aos servidores \emph{labmeme1}, \emph{jabba} e \emph{leia} usaremos a
sintaxe básica. Os servidores do \textbf{LabMeMe} estão configurados no
endereço:

\begin{verbatim}
labmeme1: 143.107.246.251
jabba: 143.107.246.253
leia: 192.168.0.1 (ip interno, acesso via jabba)
\end{verbatim}

Sendo assim, para acessar o servidor \emph{jabba}, por exemplo, o
usuário deve digitar:

\begin{verbatim}
$ ssh usuario@143.107.246.253
\end{verbatim}

No primeiro acesso, uma mensagem de aviso aparecerá perguntando se você
tem certeza se quer continuar.

\begin{verbatim}
# The authenticity of host 'github.com (207.97.227.239)' can't be established.
# RSA key fingerprint is 16:27:ac:a5:76:28:2d:36:63:1b:56:4d:eb:df:a6:48.
# Are you sure you want to continue connecting (yes/no)?
\end{verbatim}

Isso acontece pois o protocolo SSH é baseado no reconhecimento entre as
chaves públicas do computador local e do computador remoto. No primeiro
acesso tanto a chave local quanto a chave remota serão armazenadas nos
dois locais, evitando que a identidade das máquinas precise ser
verificada pelo usuário em todo acesso. Por fim, basta digitar
\emph{yes} que sua senha será pedida. (\emph{Nota: ao digitar sua senha,
a maioria dos sistemas operacionais não mostrará nenhum caracter
representando sua senha, ou seja, você digitará a senha e nada vai
aparecer na tela. Não se preocupe, isso é mais uma medida de
segurança.})

\section{Rotinas Comuns}\label{rotinas-comuns}

\subsection{Armazenando os endereços de IP das
máquinas}\label{armazenando-os-enderecos-de-ip-das-maquinas}

Lembrar dos endereços numéricos de IP das máquinas é algo que não
precisa ser feito. Em sistemas UNIX, é possível atribuir nomes a esses
endereços ao editar o arquivo /etc/hosts utilizando seu editor de
preferência (nano, vi, emacs, gedit, etc.)

\begin{verbatim}
$ emacs /etc/hosts

##
# Host Database
#
# localhost is used to configure the loopback interface
# when the system is booting.  Do not change this entry.
##
127.0.0.1       localhost
255.255.255.255 broadcasthost
::1             localhost
fe80::1%lo0     localhost
\end{verbatim}

Ao editar esse arquivo, basta inserir na primeira coluna o endereço de
IP da máquina desejada e na segunda coluna o nome que você quer atribuir
à ela.

\begin{verbatim}
##
# Host Database
#
# localhost is used to configure the loopback interface
# when the system is booting.  Do not change this entry.
##
127.0.0.1       localhost
255.255.255.255 broadcasthost
::1             localhost
fe80::1%lo0     localhost
143.107.246.251 labmeme1
143.107.246.253 jabba
\end{verbatim}

Após atribuir um \emph{alias} (nome que damos a qualquer ``codinome'' de
um objeto ou função em UNIX) aos servidores, podemos nos conectar a eles
tanto através do endereço de IP quanto através de seu \emph{alias}.

\begin{verbatim}
$ ssh usuario@labmeme1
$ ssh usuario@jabba
\end{verbatim}

Neste exemplo, o mesmo nome do servidor foi atribuído ao seu , porém
isso pode ser modificado de acordo com sua vontade/necessidade.

\begin{verbatim}
##
# Host Database
#
# localhost is used to configure the loopback interface
# when the system is booting.  Do not change this entry.
##
127.0.0.1       localhost
255.255.255.255 broadcasthost
::1             localhost
fe80::1%lo0     localhost
143.107.246.251 servidor-mac
143.107.246.253 servidor-linux
\end{verbatim}

Aqui o acesso seria feito da seguinte maneira:

\begin{verbatim}
$ ssh usuario@servidor-mac
$ ssh usuario@servidor-linux
\end{verbatim}

\subsection{Acessando nós escravos}\label{acessando-nos-escravos}

Da mesma forma como você pode fazer no seu computador, o nó escravo do
nosso servidor Linux (\emph{leia}) pode ser acessado via ssh através do
nó \emph{jabba}

\begin{verbatim}
$ *dentro do nó jabba* ssh usuario@leia
\end{verbatim}

\subsection{\texorpdfstring{\emph{Tunneling} por
ssh}{Tunneling por ssh}}\label{tunneling-por-ssh}

Em alguns casos, como no serviço de computação em nuvem da USP, para
maior segurança o acesso aos servidores não é feito de forma direta.
Nesse caso, o acesso é inicialmente feito a uma máquina (virtual nesse
caso, podendo ser física) que media a ``conversa'' entre a máquina local
e os servidores de processamento. Sendo assim, o acesso aos servidores
de processamento deve ser feito em duas etapas: 1. Acesso via ssh:
\textbf{ssh usuario@shark.lcca.usp.br}; 2. Acesso via ssh a partir dessa
máquina aos servidores \emph{aguia} e \emph{jaguar}: \textbf{ssh
usuario@aguia} ou \textbf{ssh usuario@jaguar}.

Sendo assim, a cópia de arquivos de/para o servidor (capítulo 2)
precisaria também ser feita seguindo essas duas etapas. Porém, as
máquinas intermediárias costumam ter limitação de recursos
(principalmente de armazenamento), além de ser um processo bastante
inconveniente copiar arquivos para um determinado lugar para em seguida
copiá-los novamente para uma nova localidade. Para evitar esse tipo de
problemas, é possível estabelecer uma conexão de \emph{tunneling} por
ssh, que cria literalmente um túnel de acesso direto entre a máquina
local e os servidores de processamento. Essa conexão é criada da
seguinte forma (usando uma conexão com o servidor \emph{aguia} como
exemplo:

\begin{verbatim}
$ ssh -2 -L 8020:aguia.lcca.usp.br:22 usuario@shark.lcca.usp.br
\end{verbatim}

onde \textbf{-L} indica que o link entre sua máquina e o servidor deve
ser feito através da porta 8020 (a porta 22 é a porta padrão de
comunicação via ssh). Essa conexão deve ser mantida aberta para que os
arquivos possam ser copiados diretamente da máquina local para os
servidores de processamento e vice-versa.

\subsection{Encerrando uma conexão}\label{encerrando-uma-conexao}

Para encerrar uma conexão ssh, basta digitar \textbf{exit} ou apertar as
teclas \textbf{Ctrl+d}

\section{\texorpdfstring{Copiando arquivos \emph{via}
ssh}{Copiando arquivos via ssh}}\label{copiando-arquivos-via-ssh}

A rotina de trabalho em servidores remotos envolve muitas etapas de
cópia de arquivos tanto do computador local para o remoto quanto no
caminho inverso. A cópia de arquivos via ssh se assemelha bastante à
cópia de arquivos localmente, porém usa-se o comando \textbf{scp} e seus
argumentos.

Os argumentos mais comumente usados são:

\begin{longtable}[c]{@{}ll@{}}
\toprule
Argumento & Função\tabularnewline
\midrule
\endhead
-r & \textbf{recursive}: necessário para copiar múltiplos arquivos e/ou
pastas\tabularnewline
-v & \textbf{verbose}: imprime mensagens de progresso\tabularnewline
-P & \textbf{port}: indica a porta pela qual os arquivos devem ser
copiados\tabularnewline
-q & \textbf{quiet}: não imprime mensagens nem avisos\tabularnewline
-l & \textbf{limit}: limita o uso de banda de conexão (em bytes) para a
cópia dos arquivos\tabularnewline
\bottomrule
\end{longtable}

\subsection{Copiando um arquivo}\label{copiando-um-arquivo}

Neste exemplo, vamos copiar o arquivo \textbf{foo.txt} do computador
local para a pasta \emph{/home} do servidor. (Para os exemplos a seguir,
assumiremos que você já domina os comandos básicos de shell).

\begin{verbatim}
$ scp foo.txt usuario@jabba:~/
\end{verbatim}

\subsection{Copiando vários arquivos ou
pasta}\label{copiando-varios-arquivos-ou-pasta}

Agora, vamos copiar a pasta \emph{phylogenies} localizada na pasta
\emph{/home/usuario/} para a pasta \emph{\textasciitilde{}/Dropbox}.

\begin{verbatim}
$ scp -r usuario@143.107.246.251:~/usuario/phylogenies ~/Dropbox
\end{verbatim}

Lembre-se de tomar cuidado com os caminhos para os arquivos. Verifique
se você está na pasta desejada para usar a forma simplificada de
caminho, ou certifique-se de passar o caminho completo do(s)
arquivo(s)/pasta(s) a ser(em) copiados(as).

\subsection{\texorpdfstring{Copiando arquivos usando o
\emph{tunneling}}{Copiando arquivos usando o tunneling}}\label{copiando-arquivos-usando-o-tunneling}

Como mencionado anteriormente, em servidores como o da nuvem da USP
podemos copiar arquivos diretamente para os servidores de processamento
sem ter que copiar primeiro os arquivos para o gatekeeper e em seguida
copiar de lá para os servidores. Para isso, o primeiro passo é ativar a
conexão de \emph{tunneling} como indicado na seção \textbf{2.3}.

Em seguida, utilizando o protocolo scp, podemos copiar um ou mais
arquivos ou pastas diretamente para os servidores. IMPORTANTE: lembrar a
porta usada no \emph{tunneling}! Ela será necessária nesta etapa.

\begin{verbatim}
$ scp -r -P 8020 ~/Dropbox/phylogenies usuario@localhost:~/projects
\end{verbatim}

Este comando copiou a pasta de filogenias diretamente para o servidor ao
qual nos conectamos usando a porta 8020 que foi determinada na conexão
de \emph{tunneling}. O comando para copiar arquivos do servidor para o
computador local é similar:

\begin{verbatim}
$ scp -r -P 8020 usuario@localhost:~/projects/phylogenies ~/backup/
\end{verbatim}

\end{document}
