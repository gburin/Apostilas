\documentclass[]{article}
\usepackage{lmodern}
\usepackage{amssymb,amsmath}
\usepackage{ifxetex,ifluatex}
\usepackage{fixltx2e} % provides \textsubscript
\ifnum 0\ifxetex 1\fi\ifluatex 1\fi=0 % if pdftex
  \usepackage[T1]{fontenc}
  \usepackage[utf8]{inputenc}
\else % if luatex or xelatex
  \ifxetex
    \usepackage{mathspec}
    \usepackage{xltxtra,xunicode}
  \else
    \usepackage{fontspec}
  \fi
  \defaultfontfeatures{Mapping=tex-text,Scale=MatchLowercase}
  \newcommand{\euro}{€}
\fi
% use upquote if available, for straight quotes in verbatim environments
\IfFileExists{upquote.sty}{\usepackage{upquote}}{}
% use microtype if available
\IfFileExists{microtype.sty}{%
\usepackage{microtype}
\UseMicrotypeSet[protrusion]{basicmath} % disable protrusion for tt fonts
}{}
\usepackage[margin=1in]{geometry}
\usepackage{color}
\usepackage{fancyvrb}
\newcommand{\VerbBar}{|}
\newcommand{\VERB}{\Verb[commandchars=\\\{\}]}
\DefineVerbatimEnvironment{Highlighting}{Verbatim}{commandchars=\\\{\}}
% Add ',fontsize=\small' for more characters per line
\usepackage{framed}
\definecolor{shadecolor}{RGB}{248,248,248}
\newenvironment{Shaded}{\begin{snugshade}}{\end{snugshade}}
\newcommand{\KeywordTok}[1]{\textcolor[rgb]{0.13,0.29,0.53}{\textbf{{#1}}}}
\newcommand{\DataTypeTok}[1]{\textcolor[rgb]{0.13,0.29,0.53}{{#1}}}
\newcommand{\DecValTok}[1]{\textcolor[rgb]{0.00,0.00,0.81}{{#1}}}
\newcommand{\BaseNTok}[1]{\textcolor[rgb]{0.00,0.00,0.81}{{#1}}}
\newcommand{\FloatTok}[1]{\textcolor[rgb]{0.00,0.00,0.81}{{#1}}}
\newcommand{\CharTok}[1]{\textcolor[rgb]{0.31,0.60,0.02}{{#1}}}
\newcommand{\StringTok}[1]{\textcolor[rgb]{0.31,0.60,0.02}{{#1}}}
\newcommand{\CommentTok}[1]{\textcolor[rgb]{0.56,0.35,0.01}{\textit{{#1}}}}
\newcommand{\OtherTok}[1]{\textcolor[rgb]{0.56,0.35,0.01}{{#1}}}
\newcommand{\AlertTok}[1]{\textcolor[rgb]{0.94,0.16,0.16}{{#1}}}
\newcommand{\FunctionTok}[1]{\textcolor[rgb]{0.00,0.00,0.00}{{#1}}}
\newcommand{\RegionMarkerTok}[1]{{#1}}
\newcommand{\ErrorTok}[1]{\textbf{{#1}}}
\newcommand{\NormalTok}[1]{{#1}}
\usepackage{graphicx}
\makeatletter
\def\maxwidth{\ifdim\Gin@nat@width>\linewidth\linewidth\else\Gin@nat@width\fi}
\def\maxheight{\ifdim\Gin@nat@height>\textheight\textheight\else\Gin@nat@height\fi}
\makeatother
% Scale images if necessary, so that they will not overflow the page
% margins by default, and it is still possible to overwrite the defaults
% using explicit options in \includegraphics[width, height, ...]{}
\setkeys{Gin}{width=\maxwidth,height=\maxheight,keepaspectratio}
\ifxetex
  \usepackage[setpagesize=false, % page size defined by xetex
              unicode=false, % unicode breaks when used with xetex
              xetex]{hyperref}
\else
  \usepackage[unicode=true]{hyperref}
\fi
\hypersetup{breaklinks=true,
            bookmarks=true,
            pdfauthor={Laboratório de Macroevolução e Macroecologia},
            pdftitle={Tutorial ggplot2},
            colorlinks=true,
            citecolor=blue,
            urlcolor=blue,
            linkcolor=magenta,
            pdfborder={0 0 0}}
\urlstyle{same}  % don't use monospace font for urls
\usepackage[normalem]{ulem}
% avoid problems with \sout in headers with hyperref:
\pdfstringdefDisableCommands{\renewcommand{\sout}{}}
\setlength{\parindent}{0pt}
\setlength{\parskip}{6pt plus 2pt minus 1pt}
\setlength{\emergencystretch}{3em}  % prevent overfull lines
\setcounter{secnumdepth}{5}

%%% Use protect on footnotes to avoid problems with footnotes in titles
\let\rmarkdownfootnote\footnote%
\def\footnote{\protect\rmarkdownfootnote}

%%% Change title format to be more compact
\usepackage{titling}
\setlength{\droptitle}{-2em}
  \title{Tutorial ggplot2}
  \pretitle{\vspace{\droptitle}\centering\huge}
  \posttitle{\par}
  \author{Laboratório de Macroevolução e Macroecologia}
  \preauthor{\centering\large\emph}
  \postauthor{\par}
  \predate{\centering\large\emph}
  \postdate{\par}
  \date{Maio/2015}




\begin{document}

\maketitle


{
\hypersetup{linkcolor=black}
\setcounter{tocdepth}{3}
\tableofcontents
}
\newpage

\section{O pacote ggplot2}\label{o-pacote-ggplot2}

O pacote \emph{ggplot2} foi desenvolvido pelo Dr.~Hadley Wickham (que
desenvolveu diversos outros pacotes importantes para \textbf{R} como o
\emph{(d)plyr} por exemplo. Este pacote implementa uma nova maneira de
criar gráficos a partir dos dados, trazendo o conceito de camadas
(\emph{layers}) para a sintaxe do \textbf{R}. Com esse novo conceito,
surge também a necessidade de se organizar os dados de uma maneira que
facilite a utilização da nova sintaxe. Para isso, os outros pacotes
(como os já mencionados \emph{(d)plyr}) foram desenvolvidos pelo
Dr.~Wickham de maneira a tornar a manipulação de dados mais ágil,
intuitiva e já adequada para o uso conjunto com \emph{ggplot2}.

\subsection{Instalação}\label{instalacao}

O pacote deve ser instalado como qualquer outro pacote no \textbf{R}:

\begin{Shaded}
\begin{Highlighting}[]
\KeywordTok{install.packages}\NormalTok{(}\StringTok{"ggplot2"}\NormalTok{)}
\end{Highlighting}
\end{Shaded}

É possível também instalar as versões mais recentes do pacote (mas que
podem apresentar instabilidades ou serem incompletas) através da função
\emph{install\_github} do pacote \textbf{devtools}. Este tipo de
instalação usa os arquivos disponíveis no repositório público do
\emph{ggplot2} no github.

\begin{Shaded}
\begin{Highlighting}[]
\KeywordTok{install.packages}\NormalTok{(}\StringTok{"devtools"}\NormalTok{)}
\KeywordTok{library}\NormalTok{(devtools)}
\KeywordTok{install_github}\NormalTok{(}\StringTok{"hadley/ggplot2"}\NormalTok{)}
\end{Highlighting}
\end{Shaded}

Após a instalação, carregue o pacote para utilização.

\begin{Shaded}
\begin{Highlighting}[]
\KeywordTok{library}\NormalTok{(ggplot2)}
\end{Highlighting}
\end{Shaded}

\section{\texorpdfstring{Como organizar os dados para o
\emph{ggplot2}}{Como organizar os dados para o ggplot2}}\label{como-organizar-os-dados-para-o-ggplot2}

É bastante comum usuários enfrentarem problemas nas primeiras vezes que
usam o \emph{ggplot2} pois ao organizarem os dados para serem importados
no \textbf{R}, procuram o fazer da maneira mais sintética possivel,
agrupando variáveis por exemplo. A tabela a seguir é um exemplo de como
\textbf{NÃO} se deve organizar os dados para serem usados com o
\emph{ggplot2} (\sout{para nenhuma situação na verdade}).

\begin{verbatim}
##            Tratamento  Valor
## 1   Tratamento A - 2h 10.424
## 2   Tratamento A - 4h  9.663
## 3   Tratamento A - 8h  9.317
## 4  Tratamento A - 16h 10.112
## 5  Tratamento A - 24h  9.219
## 6   Tratamento B - 2h  9.569
## 7   Tratamento B - 4h  9.668
## 8   Tratamento B - 8h 10.177
## 9  Tratamento B - 16h 10.388
## 10 Tratamento B - 24h  8.774
\end{verbatim}

Os objetos de entrada nas funções do \emph{ggplot2} são da classe
\emph{data.frame}. No entanto, organizar o \emph{data.frame} de maneira
a otimizar o trabalho poupa bastante tempo \sout{e dor de cabeça}. O
\emph{data.frame} a ser usado em um trabalho deve idealmente conter cada
uma das variáveis em colunas separadas, e cada observação em linhas
separadas. Assim, o exemplo anterior deveria ser idealmente organizado
da seguinte forma:

\begin{verbatim}
##     Valor Tempo   Tratamento
## 1  10.424     2 Tratamento A
## 2   9.663     4 Tratamento A
## 3   9.317     8 Tratamento A
## 4  10.112    16 Tratamento A
## 5   9.219    24 Tratamento A
## 6   9.569     2 Tratamento B
## 7   9.668     4 Tratamento B
## 8  10.177     8 Tratamento B
## 9  10.388    16 Tratamento B
## 10  8.774    24 Tratamento B
\end{verbatim}

Alguns pacotes foram desenvolvidos com o propósito de facilitar a
organização dos dados de maneira limpa e otimizada. Dentre esses
pacotes, encontram-se \emph{(d)plyr}, \emph{tidyr}, \emph{magrittr}
dentre outros.

\section{Estrutura básica de um
gráfico}\label{estrutura-basica-de-um-grafico}

Como dito na introdução, um gráfico feito no \emph{ggplot2} consiste de
diversas camadas que são adicionadas a uma estrutura básica. Essa
estrutura básica é criada usando as funções \emph{ggplot} ou
\emph{qplot}. Essas funções possuem dois componentes essenciais:
\emph{data}, que como o nome diz recebe o \emph{data.frame} a ser
utilizado para a construção do gráfico, e \emph{mapping}, que é o
argumento que controla a estética do gráfico, ou seja, que controla
quais variáveis serão plotadas nos eixos \emph{x} e \emph{y}, dentre
outros detalhes que serão abordados posteriormente. Para o mapeamento
dos dados, usaremos a função \emph{aes}, na qual indicaremos qual
variável deve ser posicionada em cada eixo, além de outros argumentos
relativos a cor, tamanho, etc.

Diogo Melo, do Laboratório de Evolução de Mamíferos do IB-USP sintetizou
uma regra geral para gráficos no \emph{ggplot2}:

\begin{Shaded}
\begin{Highlighting}[]
\KeywordTok{ggplot}\NormalTok{(data_frame_entrada, }\KeywordTok{aes}\NormalTok{(}\DataTypeTok{x =} \NormalTok{coluna_eixo_x,}
                               \DataTypeTok{y =} \NormalTok{coluna_eixo_y,}
                               \DataTypeTok{group =} \NormalTok{coluna_agrupadora,}
                               \DataTypeTok{color =} \NormalTok{coluna_das_cores))}
\NormalTok{+}\StringTok{ }\KeywordTok{geom_tipo_do_grafico}\NormalTok{(opções que não dependem dos dados,}
                       \KeywordTok{aes}\NormalTok{(opções que dependem))}
\end{Highlighting}
\end{Shaded}

Para exemplificar a construção de um gráfico, usaremos o conjunto de
dados do próprio pacote \emph{ggplot2} chamado \textbf{diamonds}.

\begin{Shaded}
\begin{Highlighting}[]
\KeywordTok{data}\NormalTok{(diamonds)}
\KeywordTok{head}\NormalTok{(diamonds)}
\end{Highlighting}
\end{Shaded}

\begin{verbatim}
##   carat       cut color clarity depth table price    x    y    z
## 1  0.23     Ideal     E     SI2  61.5    55   326 3.95 3.98 2.43
## 2  0.21   Premium     E     SI1  59.8    61   326 3.89 3.84 2.31
## 3  0.23      Good     E     VS1  56.9    65   327 4.05 4.07 2.31
## 4  0.29   Premium     I     VS2  62.4    58   334 4.20 4.23 2.63
## 5  0.31      Good     J     SI2  63.3    58   335 4.34 4.35 2.75
## 6  0.24 Very Good     J    VVS2  62.8    57   336 3.94 3.96 2.48
\end{verbatim}

\subsection{Gráficos unidimensionais (boxplots e
histogramas)}\label{graficos-unidimensionais-boxplots-e-histogramas}

Um bom modo de iniciar qualquer análise é fazendo uma análise
exploratória dos dados. Para isso, alguns tipos específicos de gráficos
como \emph{boxplots} ou histogramas são maneiras simples de visualizar a
estrutura básica dos dados. Estes tipos de gráfico são ditos
unidimensionais, pois representam apenas uma variável. Neste tópico,
veremos como criar esses dois gráficos de maneira simples (complementos
serão adicionados mais para frente no tutorial).

O primeiro passo para criarmos um gráfico é então gerar o objeto da
classe \textbf{ggplot}. \textbf{DICA: armazene as camadas básicas do seu
gráfico em um objeto, pois isso facilita a adição de novas camadas sem a
necessidade de repetir o código inteiro.}

\begin{Shaded}
\begin{Highlighting}[]
\NormalTok{g1 <-}\StringTok{ }\KeywordTok{ggplot}\NormalTok{(}\DataTypeTok{data=}\NormalTok{diamonds,}\KeywordTok{aes}\NormalTok{(}\DataTypeTok{x=}\DecValTok{1}\NormalTok{,}\DataTypeTok{y=}\NormalTok{price))}
\end{Highlighting}
\end{Shaded}

No exemplo acima, indicamos que a variável \emph{price} deve ser
atribuída ao eixo \emph{y}, e o eixo \emph{x} recebe o valor 1 apenas
para posicionar o gráfico. Além disso o código que acabamos de rodar
serve apenas para criar um objeto da classe \textbf{ggplot}, porém ele
ainda não produz nenhum gráfico. Ao chamarmos o objeto g1:

\begin{Shaded}
\begin{Highlighting}[]
\NormalTok{## Error: No layers in plot}
\end{Highlighting}
\end{Shaded}

Isso se deve ao fato de que nenhuma camada de gráfico foi adicionada, ou
seja, não especificamos qual o tipo de gráfico desejamos produzir. A
maioria das camadas iniciais podem ser criadas usando as funções da
família \textbf{geom\_}. No primeiro exemplo, vamos criar um boxplot
para termos uma ideia de valores mínimos, máximos, média e posição dos
quartis.

\begin{Shaded}
\begin{Highlighting}[]
\NormalTok{g1 +}\StringTok{ }\KeywordTok{geom_boxplot}\NormalTok{()}
\end{Highlighting}
\end{Shaded}

\includegraphics{ggplot2_files/figure-latex/unnamed-chunk-9.pdf}

Note a sintaxe bastante intuitiva que usa o símbolo ``+'' para adicionar
camadas. Isso torna o \emph{ggplot2} um pacote bastante prático
\sout{depois de pegar o jeito} para criar gráficos informativos, bonitos
e prontos para publicação.

Para criarmos um histograma simples, é igualmente fácil:

\begin{Shaded}
\begin{Highlighting}[]
\KeywordTok{ggplot}\NormalTok{(diamonds,}\KeywordTok{aes}\NormalTok{(}\DataTypeTok{x=}\NormalTok{price)) +}\StringTok{ }\KeywordTok{geom_histogram}\NormalTok{()}
\end{Highlighting}
\end{Shaded}

\begin{verbatim}
## stat_bin: binwidth defaulted to range/30. Use 'binwidth = x' to adjust this.
\end{verbatim}

\includegraphics{ggplot2_files/figure-latex/unnamed-chunk-10.pdf}

Observe no entanto que para o histograma a variável de interesse está no
eixo \emph{x}, já que o eixo \emph{y} contém a contagem de valores em
cada intervalo.

Os gráficos básicos podem não parecer tão bonitos assim à primeira
vista; vamos aprender em seções posteriores como modificar a estética
dos gráficos (cores, tamanho, bordas, plano de fundo, etc.).

\subsection{Gráficos bidimensionais (gráficos de dispersão, modelos
lineares,
etc.)}\label{graficos-bidimensionais-graficos-de-dispersao-modelos-lineares-etc.}

\subsubsection{Gráficos de dispersão}\label{graficos-de-dispersao}

Em grande parte dos estudos em ecologia buscamos analisar como duas ou
mais variáveis se comportam umas em relação às outras. Para isso, um bom
ponto de partida é construir gráficos de dispersão
(\emph{scatterplots}). No exemplo a seguir construiremos um gráfico de
dispersão entre as variáveis \emph{carat} e \emph{price}.

\begin{Shaded}
\begin{Highlighting}[]
\NormalTok{g2 <-}\StringTok{ }\KeywordTok{ggplot}\NormalTok{(diamonds,}\KeywordTok{aes}\NormalTok{(}\DataTypeTok{x=}\NormalTok{carat,}\DataTypeTok{y=}\NormalTok{price))}
\NormalTok{g2 +}\StringTok{ }\KeywordTok{geom_point}\NormalTok{()}
\end{Highlighting}
\end{Shaded}

\includegraphics{ggplot2_files/figure-latex/unnamed-chunk-11.pdf}

Note que no exemplo acima, estamos indicando para a função ggplot que as
camadas que forem adicionadas de agora em diante devem conter os valores
de \emph{carat} no eixo \emph{x} e de \emph{price} no eixo \emph{y}.
Isso pode ser modificado posteriormente (como nos casos onde você quer
por exemplo acrescentar uma outra variável no eixo \emph{y}), e será
exemplificado mais adiante. No entanto, a maioria (se não todas) das
funções da família \textbf{geom\_} também possuem os argumentos básicos
\emph{data} e \emph{mapping}. Sendo assim, uma outra forma de criar o
mesmo gráfico é:

\begin{Shaded}
\begin{Highlighting}[]
\KeywordTok{ggplot}\NormalTok{() +}\StringTok{ }\KeywordTok{geom_point}\NormalTok{(}\DataTypeTok{data=}\NormalTok{diamonds,}\DataTypeTok{mapping=}\KeywordTok{aes}\NormalTok{(}\DataTypeTok{x=}\NormalTok{carat,}\DataTypeTok{y=}\NormalTok{price))}
\end{Highlighting}
\end{Shaded}

\includegraphics{ggplot2_files/figure-latex/unnamed-chunk-12.pdf}

Caso a cor padrão \sout{seja feia} não seja adequada, podemos indicar
para o gráfico qual cor queremos usar para os pontos. Como cor é um dos
componentes essenciais da parte estética de um gráfico, essa
especificação deve entrar na função \emph{aes}.

\begin{Shaded}
\begin{Highlighting}[]
\NormalTok{g2 +}\StringTok{ }\KeywordTok{geom_point}\NormalTok{(}\DataTypeTok{mapping=}\KeywordTok{aes}\NormalTok{(}\DataTypeTok{colour=}\StringTok{"red"}\NormalTok{))}
\end{Highlighting}
\end{Shaded}

\includegraphics{ggplot2_files/figure-latex/unnamed-chunk-13.pdf}

Ao analisarmos o conteúdo do objeto \textbf{diamonds} notamos que os
valores de \emph{carat} e \emph{price} são divididos por classe de
\emph{cut}, \emph{color} e \emph{clarity}. Assim, podemos por exemplo
representar o mesmo conjunto de dados separados por classe de
\emph{clarity} por exemplo. Se seu objeto contendo os dados estiver mal
organizado, você levaria um tempo razoável para adicionar uma cada para
cada classe cada uma com uma cor diferente. Porém, como já aprendemos a
organizar nossos dados de maneira idea, podemos facilmente usar a
variável \emph{clarity} para atribuir diferentes cores que representarão
cada uma das classes da seguinte forma:

\begin{Shaded}
\begin{Highlighting}[]
\NormalTok{g2 +}\StringTok{ }\KeywordTok{geom_point}\NormalTok{(}\DataTypeTok{mapping=}\KeywordTok{aes}\NormalTok{(}\DataTypeTok{colour=}\NormalTok{clarity))}
\end{Highlighting}
\end{Shaded}

\includegraphics{ggplot2_files/figure-latex/unnamed-chunk-14.pdf}

Essa separação em classes de \emph{clarity} pode ser feita em outros
tipos de gráfico também (os \emph{outliers} foram removidos para fins de
visualização):

\begin{Shaded}
\begin{Highlighting}[]
\KeywordTok{ggplot}\NormalTok{(}\DataTypeTok{data=}\NormalTok{diamonds,}
       \KeywordTok{aes}\NormalTok{(}\DataTypeTok{x=}\NormalTok{clarity, }\DataTypeTok{y=}\NormalTok{price, }\DataTypeTok{group=}\KeywordTok{interaction}\NormalTok{(clarity,cut), }\DataTypeTok{color=}\NormalTok{clarity)) +}
\StringTok{           }\KeywordTok{geom_boxplot}\NormalTok{(}\DataTypeTok{outlier.shape=}\OtherTok{NA}\NormalTok{)}
\end{Highlighting}
\end{Shaded}

\includegraphics{ggplot2_files/figure-latex/unnamed-chunk-15.pdf}

\subsubsection{Adicionando linhas de
tendência}\label{adicionando-linhas-de-tendencia}

Os gráficos de dispersão que construímos até agora indicam que existe
uma relação positiva entre quilates e preço. Nesse caso, podemos ajustar
um modelo linear que indique como se dá essa correlação. Para isso, o
pacote \emph{ggplot2} possui funções específicas para algumas
estatísticas comumente usadas. Aqui, usaremos a função
\textbf{stat\_smooth} para isso.

\begin{Shaded}
\begin{Highlighting}[]
\NormalTok{g2 +}\StringTok{ }\KeywordTok{geom_point}\NormalTok{(}\KeywordTok{aes}\NormalTok{(}\DataTypeTok{color=}\NormalTok{clarity)) +}\StringTok{ }\KeywordTok{geom_smooth}\NormalTok{(}\DataTypeTok{method =} \StringTok{"lm"}\NormalTok{, }\DataTypeTok{se=}\OtherTok{FALSE}\NormalTok{, }\KeywordTok{aes}\NormalTok{(}\DataTypeTok{color=}\StringTok{"red"}\NormalTok{))}
\end{Highlighting}
\end{Shaded}

\includegraphics{ggplot2_files/figure-latex/unnamed-chunk-16.pdf}

A linha de regressão adicionada no gráfico anterior corresponde à
correlação entre \emph{carat} e \emph{price} indepentende da classe de
\emph{clarity}. Porém, dependendo do caso pode ser mais interessante
criar uma reta de correlação para cada classe de \emph{clarity}. Fazemos
isso da seguinte forma (adicionando também intervalos de confiança da
média):

\begin{Shaded}
\begin{Highlighting}[]
\NormalTok{g3 <-}\StringTok{ }\NormalTok{g2 +}\StringTok{ }\KeywordTok{geom_point}\NormalTok{(}\KeywordTok{aes}\NormalTok{(}\DataTypeTok{color=}\NormalTok{clarity)) +}\StringTok{ }\KeywordTok{geom_smooth}\NormalTok{(}\DataTypeTok{method =} \StringTok{"lm"}\NormalTok{, }\DataTypeTok{se=}\OtherTok{TRUE}\NormalTok{, }\KeywordTok{aes}\NormalTok{(}\DataTypeTok{color=}\NormalTok{clarity))}
\NormalTok{g3}
\end{Highlighting}
\end{Shaded}

\includegraphics{ggplot2_files/figure-latex/unnamed-chunk-17.pdf}

\newpage

\section{Personalizando os gráficos}\label{personalizando-os-graficos}

\subsection{Alterar rótulos dos eixos}\label{alterar-rotulos-dos-eixos}

\begin{Shaded}
\begin{Highlighting}[]
\NormalTok{g3 +}\StringTok{ }\KeywordTok{xlab}\NormalTok{(}\StringTok{"Quilates"}\NormalTok{) +}\StringTok{ }\KeywordTok{ylab}\NormalTok{(}\StringTok{"Preco"}\NormalTok{)}
\end{Highlighting}
\end{Shaded}

\includegraphics{ggplot2_files/figure-latex/unnamed-chunk-18.pdf}

\newpage

\subsection{Alterar limites dos eixos}\label{alterar-limites-dos-eixos}

\begin{Shaded}
\begin{Highlighting}[]
\NormalTok{g3 +}\StringTok{ }\KeywordTok{xlim}\NormalTok{(}\DecValTok{0}\NormalTok{,}\DecValTok{10}\NormalTok{) +}\StringTok{ }\KeywordTok{ylim}\NormalTok{(}\DecValTok{0}\NormalTok{,}\DecValTok{30000}\NormalTok{)}
\end{Highlighting}
\end{Shaded}

\includegraphics{ggplot2_files/figure-latex/unnamed-chunk-19.pdf}

\newpage

\subsection{Alterar tamanho e direção das fontes dos
eixos}\label{alterar-tamanho-e-direcao-das-fontes-dos-eixos}

\begin{Shaded}
\begin{Highlighting}[]
\NormalTok{g3 +}\StringTok{ }\KeywordTok{theme}\NormalTok{(}\DataTypeTok{axis.text.x =} \KeywordTok{element_text}\NormalTok{(}\DataTypeTok{angle=}\DecValTok{45}\NormalTok{,}\DataTypeTok{vjust=}\DecValTok{1}\NormalTok{,}\DataTypeTok{size=}\DecValTok{30}\NormalTok{))}
\end{Highlighting}
\end{Shaded}

\includegraphics{ggplot2_files/figure-latex/unnamed-chunk-20.pdf}

\newpage

\subsection{Inserir anotações}\label{inserir-anotacoes}

\begin{Shaded}
\begin{Highlighting}[]
\NormalTok{g3 +}\StringTok{ }\KeywordTok{xlim}\NormalTok{(}\DecValTok{0}\NormalTok{,}\DecValTok{10}\NormalTok{) +}\StringTok{ }\KeywordTok{ylim}\NormalTok{(}\DecValTok{0}\NormalTok{,}\DecValTok{30000}\NormalTok{) +}\StringTok{ }\KeywordTok{annotate}\NormalTok{(}\StringTok{"text"}\NormalTok{,}\DataTypeTok{x=}\FloatTok{9.5}\NormalTok{,}\DataTypeTok{y=}\DecValTok{30000}\NormalTok{,}\DataTypeTok{label=} \StringTok{"Painel a"}\NormalTok{,}\DataTypeTok{color=}\StringTok{"red"}\NormalTok{)}
\end{Highlighting}
\end{Shaded}

\includegraphics{ggplot2_files/figure-latex/unnamed-chunk-21.pdf}

\newpage

\begin{Shaded}
\begin{Highlighting}[]
\NormalTok{g3 +}\StringTok{ }\KeywordTok{xlim}\NormalTok{(}\DecValTok{0}\NormalTok{,}\DecValTok{10}\NormalTok{) +}\StringTok{ }\KeywordTok{ylim}\NormalTok{(}\DecValTok{0}\NormalTok{,}\DecValTok{30000}\NormalTok{) +}
\StringTok{    }\KeywordTok{annotate}\NormalTok{(}\StringTok{"rect"}\NormalTok{,}\DataTypeTok{xmin=}\DecValTok{3}\NormalTok{, }\DataTypeTok{xmax=}\DecValTok{4}\NormalTok{, }\DataTypeTok{ymin=}\DecValTok{10000}\NormalTok{,}
             \DataTypeTok{ymax=}\DecValTok{20000}\NormalTok{, }\DataTypeTok{alpha=}\NormalTok{.}\DecValTok{2}\NormalTok{, }\DataTypeTok{color=}\StringTok{"blue"}\NormalTok{,}\DataTypeTok{fill=}\StringTok{"green"}\NormalTok{)}
\end{Highlighting}
\end{Shaded}

\includegraphics{ggplot2_files/figure-latex/unnamed-chunk-22.pdf}

\newpage

\begin{Shaded}
\begin{Highlighting}[]
\NormalTok{g3 +}\StringTok{ }\KeywordTok{xlim}\NormalTok{(}\DecValTok{0}\NormalTok{,}\DecValTok{10}\NormalTok{) +}\StringTok{ }\KeywordTok{ylim}\NormalTok{(}\DecValTok{0}\NormalTok{,}\DecValTok{30000}\NormalTok{) +}
\StringTok{    }\KeywordTok{annotate}\NormalTok{(}\StringTok{"segment"}\NormalTok{, }\DataTypeTok{x=}\DecValTok{3}\NormalTok{, }\DataTypeTok{xend=}\DecValTok{4}\NormalTok{, }\DataTypeTok{y=}\DecValTok{30000}\NormalTok{, }\DataTypeTok{yend=}\DecValTok{20000}\NormalTok{,}
             \DataTypeTok{color=}\StringTok{"orange"}\NormalTok{, }\DataTypeTok{arrow=}\KeywordTok{arrow}\NormalTok{(}\DataTypeTok{length=}\KeywordTok{unit}\NormalTok{(}\DecValTok{1}\NormalTok{,}\StringTok{"cm"}\NormalTok{)))}
\end{Highlighting}
\end{Shaded}

\includegraphics{ggplot2_files/figure-latex/unnamed-chunk-23.pdf}

\newpage

\subsection{Alterar paleta de cores}\label{alterar-paleta-de-cores}

\begin{Shaded}
\begin{Highlighting}[]
\NormalTok{g2 +}\StringTok{ }\KeywordTok{geom_point}\NormalTok{(}\KeywordTok{aes}\NormalTok{(}\DataTypeTok{color=}\NormalTok{clarity)) +}\StringTok{ }\KeywordTok{scale_color_brewer}\NormalTok{(}\DataTypeTok{type=}\StringTok{"seq"}\NormalTok{,}\DataTypeTok{palette=}\StringTok{"Set3"}\NormalTok{)}
\end{Highlighting}
\end{Shaded}

\includegraphics{ggplot2_files/figure-latex/unnamed-chunk-24.pdf}

\newpage

\subsection{Separar gráficos em
painéis}\label{separar-graficos-em-paineis}

\begin{Shaded}
\begin{Highlighting}[]
\NormalTok{g2 +}\StringTok{ }\KeywordTok{geom_point}\NormalTok{(}\KeywordTok{aes}\NormalTok{(}\DataTypeTok{color=}\NormalTok{clarity)) +}\StringTok{ }\KeywordTok{facet_grid}\NormalTok{(. ~}\StringTok{ }\NormalTok{cut)}
\end{Highlighting}
\end{Shaded}

\includegraphics{ggplot2_files/figure-latex/unnamed-chunk-25.pdf}

\newpage

\begin{Shaded}
\begin{Highlighting}[]
\NormalTok{g2 +}\StringTok{ }\KeywordTok{geom_point}\NormalTok{(}\KeywordTok{aes}\NormalTok{(}\DataTypeTok{color=}\NormalTok{clarity)) +}\StringTok{ }\KeywordTok{facet_grid}\NormalTok{(cut ~}\StringTok{ }\NormalTok{.)}
\end{Highlighting}
\end{Shaded}

\includegraphics{ggplot2_files/figure-latex/unnamed-chunk-26.pdf}

\newpage

\begin{Shaded}
\begin{Highlighting}[]
\NormalTok{g2 +}\StringTok{ }\KeywordTok{geom_point}\NormalTok{(}\KeywordTok{aes}\NormalTok{(}\DataTypeTok{color=}\NormalTok{clarity)) +}\StringTok{ }\KeywordTok{facet_grid}\NormalTok{(cut ~}\StringTok{ }\NormalTok{clarity)}
\end{Highlighting}
\end{Shaded}

\includegraphics{ggplot2_files/figure-latex/unnamed-chunk-27.pdf}

\newpage

\subsection{Alterar posição, título e tamanho da
legenda}\label{alterar-posicao-titulo-e-tamanho-da-legenda}

\begin{Shaded}
\begin{Highlighting}[]
\NormalTok{g3 +}\StringTok{ }\KeywordTok{xlim}\NormalTok{(}\DecValTok{0}\NormalTok{,}\DecValTok{10}\NormalTok{) +}\StringTok{ }\KeywordTok{ylim}\NormalTok{(}\DecValTok{0}\NormalTok{,}\DecValTok{30000}\NormalTok{) +}\StringTok{ }\KeywordTok{theme}\NormalTok{(}\DataTypeTok{legend.position =} \StringTok{"none"}\NormalTok{)}
\end{Highlighting}
\end{Shaded}

\includegraphics{ggplot2_files/figure-latex/unnamed-chunk-28.pdf}

\newpage

\begin{Shaded}
\begin{Highlighting}[]
\NormalTok{g3 +}\StringTok{ }\KeywordTok{xlim}\NormalTok{(}\DecValTok{0}\NormalTok{,}\DecValTok{10}\NormalTok{) +}\StringTok{ }\KeywordTok{ylim}\NormalTok{(}\DecValTok{0}\NormalTok{,}\DecValTok{30000}\NormalTok{) +}\StringTok{ }\KeywordTok{theme}\NormalTok{(}\DataTypeTok{legend.position =} \StringTok{"bottom"}\NormalTok{)}
\end{Highlighting}
\end{Shaded}

\includegraphics{ggplot2_files/figure-latex/unnamed-chunk-29.pdf}

\newpage

\begin{Shaded}
\begin{Highlighting}[]
\NormalTok{g3 +}\StringTok{ }\KeywordTok{xlim}\NormalTok{(}\DecValTok{0}\NormalTok{,}\DecValTok{10}\NormalTok{) +}\StringTok{ }\KeywordTok{ylim}\NormalTok{(}\DecValTok{0}\NormalTok{,}\DecValTok{30000}\NormalTok{) +}\StringTok{ }\KeywordTok{theme}\NormalTok{(}\DataTypeTok{legend.position =} \StringTok{"bottom"}\NormalTok{, }\DataTypeTok{legend.title =} \KeywordTok{element_blank}\NormalTok{())}
\end{Highlighting}
\end{Shaded}

\includegraphics{ggplot2_files/figure-latex/unnamed-chunk-30.pdf}

\newpage

\begin{Shaded}
\begin{Highlighting}[]
\NormalTok{g3 +}\StringTok{ }\KeywordTok{xlim}\NormalTok{(}\DecValTok{0}\NormalTok{,}\DecValTok{10}\NormalTok{) +}\StringTok{ }\KeywordTok{ylim}\NormalTok{(}\DecValTok{0}\NormalTok{,}\DecValTok{30000}\NormalTok{) +}\StringTok{ }\KeywordTok{theme}\NormalTok{(}\DataTypeTok{legend.position =} \KeywordTok{c}\NormalTok{(.}\DecValTok{8}\NormalTok{,.}\DecValTok{8}\NormalTok{))}
\end{Highlighting}
\end{Shaded}

\includegraphics{ggplot2_files/figure-latex/unnamed-chunk-31.pdf}

\newpage

\begin{Shaded}
\begin{Highlighting}[]
\NormalTok{g3 +}\StringTok{ }\KeywordTok{xlim}\NormalTok{(}\DecValTok{0}\NormalTok{,}\DecValTok{10}\NormalTok{) +}\StringTok{ }\KeywordTok{ylim}\NormalTok{(}\DecValTok{0}\NormalTok{,}\DecValTok{30000}\NormalTok{) +}\StringTok{ }\KeywordTok{theme}\NormalTok{(}\DataTypeTok{legend.position =} \KeywordTok{c}\NormalTok{(.}\DecValTok{8}\NormalTok{,.}\DecValTok{2}\NormalTok{))}
\end{Highlighting}
\end{Shaded}

\includegraphics{ggplot2_files/figure-latex/unnamed-chunk-32.pdf}

\newpage

\subsection{Alterar temas}\label{alterar-temas}

\begin{Shaded}
\begin{Highlighting}[]
\NormalTok{g3 +}\StringTok{ }\KeywordTok{theme_bw}\NormalTok{()}
\end{Highlighting}
\end{Shaded}

\includegraphics{ggplot2_files/figure-latex/unnamed-chunk-33.pdf}

\newpage

\begin{Shaded}
\begin{Highlighting}[]
\NormalTok{g3 +}\StringTok{ }\KeywordTok{theme}\NormalTok{(}\DataTypeTok{panel.grid.major =} \KeywordTok{element_line}\NormalTok{(}\DataTypeTok{colour =} \StringTok{"red"}\NormalTok{, }\DataTypeTok{linetype =} \StringTok{"dotted"}\NormalTok{), }\DataTypeTok{panel.background =} \KeywordTok{element_rect}\NormalTok{(}\DataTypeTok{fill =} \StringTok{"green"}\NormalTok{))}
\end{Highlighting}
\end{Shaded}

\includegraphics{ggplot2_files/figure-latex/unnamed-chunk-34.pdf}

\end{document}
